\PassOptionsToPackage{type=CC,modifier=by,version=4.0}{doclicense}
\documentclass[conference,a4paper,flushend]{cs-techrep} % Optional: flushend or pbalance
\pdfoutput=1 % pdflatex hint for arxiv.org (within first 5 lines)

% Class cs-techrep.cls loads biblatex / biber with predefined options
\addbibresource{embedded.bib}       % its content is declared below, embedded within this tex-file
\addbibresource{cpn_all_all.bib}    % includes all previous CyberLytics@OTH-AW technical reports

% ======================================================================
% EDIT THESE:

\cstechrepAuthorListTex{Vorname Nachname, Christoph P.\ Neumann\,\orcidlink{0000-0002-5936-631X}}
% for bibtex:
\cstechrepAuthorListBib{Vorname Nachname and Christoph P. Neumann}

% Capitalization: https://capitalizemytitle.com/style/Chicago/
\cstechrepTitleTex{The cs-techrep \textsmaller{CTAN} Package and LaTeX Class Formatting Example: TechRep in Computer Science}
 % IF you need manual linebreaks in the titel, then clone the title without linebreaks for BibTeX:
\cstechrepTitleBib{{\cstechrepTitleTex}}

\cstechrepDepartment{Department of Electrical Engineering, Media, and Computer Science}
\cstechrepInstitution{Ostbayerische Technische Hochschule Amberg\-/Weiden}
\cstechrepAddress{Amberg, Germany}
\cstechrepType{Technical Report}
\cstechrepYear{2025} % CTAN
\cstechrepMonth{03} % CTAN
\cstechrepNumber{SCWR-\cstechrepYear{}-99}
\cstechrepLang{english}  % en-US

% Special remark on babel/csquotes terminology in regard with US-vs-UK:
% en-US  = [english]/[american]/[usenglish] (+ [canadian])
% en-UK  =           [british] /[ukenglish] (+ [australian]) <OXFORD>
% For cs-techrep (like ACM), the recommended english variant is en-US!

% DO NOT DELETE THIS:
\filecontentsForceExpansion|[] % force command expansion inside a filecontents* environment
\begin{filecontents*}[overwrite]{selfref.bib}
    @TECHREPORT{selfref,
        author = {|cstechrepAuthorListBib},
        title  = {\cstechrepTitleBib},
        institution = {\cstechrepInstitution, \cstechrepDepartment},
        type   = {\cstechrepType},
        number = {\cstechrepNumber},
        year   = {|cstechrepYear},
        month  = {|cstechrepMonth},
        langid  = {|cstechrepLang},
    }
\end{filecontents*}

% ======================================================================
% EDIT THIS:

\begin{filecontents}[overwrite]{embedded.bib}
@online{ieee2015howto,
	author = {Michael Shell},
	title = {How to Use the {IEEEtran} \LaTeX\ Class},
	url = {http://mirrors.ctan.org/macros/latex/contrib/IEEEtran/IEEEtran_HOWTO.pdf},
	year = {2015},
	urldate = {2025-03-10}
}
@online{ieee2018formattingrules,
	author = {{IEEE}},
	title = {Conference Template and Formatting Specifications},
	url = {https://www.ieee.org/content/dam/ieee-org/ieee/web/org/conferences/Conference-template-A4.doc},
	year = {2018},
	urldate={2025-03-10}
}
@online{iaria2014formattingrules,
	author = {{IARIA}},
	title = {Formatting Rules},
	url = {http://www.iaria.org/formatting.doc},
	year = {2014},
	urldate={2025-03-10}
}
@online{iaria2009editorialrules,
	_author = {Cosmin Dini},
	author = {{IARIA}},
	title = {Editorial Rules},
	url = {https://www.iaria.org/editorialrules.html},
	year = {2009},
	urldate={2025-03-10}
}
@online{languagetool,
	author = {{LanguageTooler GmbH}},
	title  = {{LangueTool}},
	url    = {https://languagetool.org/overleaf},
	urldate={2025-03-10}
}
@online{overleaf,
	author = {{Digital Science UK Limited}},
	title  = {{Overleaf}},
	url    = {https://www.overleaf.com},
	urldate={2025-03-10}
}
@online{elsevier-imrad,
	author = {Angel Borja},
	title  = {11 steps to structuring a science paper editors will take seriously},
	url    = {https://www.elsevier.com/connect/11-steps-to-structuring-a-science-paper-editors-will-take-seriously},
	year   = 2011,
	urldate={2025-03-10}
}
@online{wiki:techrep,
	author = {{Wikipedia}},
	title  = {Technical report},
	url    = {https://en.wikipedia.org/wiki/Technical_report},
	urldate={2025-03-10}
}
\end{filecontents}

\usepackage{fontawesome} % i.a., \faWarning{}
\usepackage{relsize}     % i.a., \textsmaller{...}
\usepackage{lipsum}      % for blindtext
\usepackage{xurl}        % allow URL breaks at any alphanumerical character

% ======================================================================

% cf. https://ctan.org/pkg/acronym
% Usage:
% singular, within sentence       = \ac{gui}
% singular, beginning of sentence = \Ac{gui}
% plural, within sentence       = \acp{gui}
% plural, beginning of sentence = \Acp{gui}
\begin{acronym}
    \acro{gui}[GUI]{Graphical User Interface}
    \acro{ide}[IDE]{Integrated Development Environment}
\end{acronym}

% https://www.silbentrennung24.de/
% https://www.hyphenation24.com/
\hyphenation{block-chain block-chains Ethe-re-um}

\begin{document}
\selectlanguage{\cstechrepLang}

\maketitle

\begin{abstract}
This paper demonstrates an example of a technical report in computer science or software engineering, based on the \texttt{cs-techrep} LaTeX class.
The example is intended for beginners, e.\,g., undergraduate students.
It contains a basic outline template and usually fills it with dummy text, but some sections are describing the intent of the outline template and its sections.
Graphic exclamation marks highlight important remarks.
\end{abstract}

% A list of IEEE Computer Society appoved keywords can be obtained at
% http://www.computer.org/mc/keywords/keywords.htm
\begin{IEEEkeywords}
template; lorem ipsum.
\end{IEEEkeywords}

\{\,\faWarning{}The abstract does neither mention a thesis in which context a technical report is written nor an institution or any other organizational aspects.
It is a summary of the content of the technical report, thus, usually the objectives and architecture of a piece of software.
Do NOT remove the abstract, this section is mandatory.
Do NOT use special characters, symbols, or math in your title or abstract.
Do NOT use references in the abstract, avoid abbreviations or acronyms.
The abstract must look as one paragraph only.
Ideally, end the abstract with one sentence stressing out the main output of the paper.\}

\section{Introduction}

The cs-techrep formatting is adopted both from \textsmaller{IEEE} \cite{ieee2018formattingrules} and \textsmaller{IARIA} \cite{iaria2014formattingrules} styles.
The cs-techrep \LaTeX\ class is based on \textsmaller{IEEE}tran class \cite{ieee2015howto}.
In addition, be aware of the \faWarning{} \textsmaller{IARIA} editorial rules \cite{iaria2009editorialrules} that provide a compact, online-available, and beginner-friendly set of further advices.

\{\faWarning{} For beginners:
Familiarize yourself with IMRaD \cite{elsevier-imrad}, the predominant structure for organizing research articles.
Additionally, ensure you have a solid understanding of technical reports as a form of scientific literature,
at minimum, as outlined on Wikipedia \cite{wiki:techrep}.
Be cautious not to adopt an oversimplified interpretation of the TechRep term.\}

It is recommended to use a grammar tool, e.\,g., the LanguageTool \cite{languagetool} browser plugin in combination with Overleaf \cite{overleaf}.
%
The \texttt{cs-techrep.cls} is intended for \faWarning{} \texttt{pdflatex} plus \texttt{biblatex}\;/\;\texttt{biber} as TeX stack.

\lipsum[22]

\{\faWarning{} Recommendation: Introduction must end with a paragraph describing the structure of the paper!\}
The remainder of the paper is organized as follows: In Section~II, …

\section{Methods \textbar{} Related Work}
\lipsum[23]

\section{Results}
\lipsum[24]

\section{Discussion \textbar{} Evalution}
\lipsum[25]

\section{Conclusion}
\{\faWarning{} Recommendation: Last section must be a conclusion and might also provide an outlook on future work.\}
\lipsum[26]

\{\,\faWarning{} For beginners: you must NOT leave the bibliography blank. Add appropriate references to your related work.\}\\
\{\,\faWarning{} Recommendation: bibliographic \textsmaller{@ONLINE} entries should have an \texttt{urldate} property as access date.\}
% ======== References =========
\printbibliography[notcategory=selfref]

\end{document}
